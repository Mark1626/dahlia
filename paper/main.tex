%% For double-blind review submission, w/o CCS and ACM Reference (max submission space)
\documentclass[acmsmall,review,anonymous]{acmart}\settopmatter{printfolios=true,printccs=false,printacmref=false}

\acmConference[OOPSLA'19]{ACM SIGPLAN Conference on Programming Languages}{January 01--03, 2018}{New York, NY, USA}
\acmYear{2019}
\acmISBN{} % \acmISBN{978-x-xxxx-xxxx-x/YY/MM}
\acmDOI{} % \acmDOI{10.1145/nnnnnnn.nnnnnnn}
\startPage{1}
\setcopyright{none}
\bibliographystyle{ACM-Reference-Format}

\usepackage{pervasives}

\begin{document}

\title[]{\sys{}: Safe High-level Synthesis}

\author{Rachit Nigam}
\affiliation{\institution{Cornell University}\country{U.S.A.}}
\email{rnigam@cs.cornell.edu}
\author{Sachille Atapattu}
\affiliation{\institution{Cornell University}\country{U.S.A.} }
\email{first2.last2@inst2b.org}
\author{Theodore Bauer}
\affiliation{\institution{Cornell University}\country{U.S.A.} }
\email{first2.last2@inst2b.org}
\author{Adrian Sampson}
\affiliation{\institution{Cornell University}\country{U.S.A.} }
\email{first2.last2@inst2b.org}


%% Abstract
%% Note: \begin{abstract}...\end{abstract} environment must come
%% before \maketitle command
\begin{abstract}
With the end of Moore's law in sight, FPGAs and reconfigurable hardware are
gaining prominence in a variety of computationally heavy domains such as
machine learning. However, writing hardware programs remains challenging
because (1) the abstractions provided by hardware design languages (HDLs) is
too low level and (2) high-level synthesis (HLS) tools, which promise to
raise the level of abstraction, provide ad hoc mechanisms on top of unsafe
languages like C and C++ and repurpose wildly inefficient EDA flows. Towards
the goal of creating a safe and expressive abstractions for hardware
programming, we present \sys{}, a programming language that exposes the
unique resource constraints in hardware and guarantees hardware
realizability. This is in contrast to contemporary HLS tools which try, and
fail, at transparently compiling C/C++ programs to hardware. HLS tools often
add inefficient hardware in order to compile these programs and make it
significantly harder to understand the performance characteristics of the
generated hardware. Instead of hiding hardware constraints, \sys{} exposes
them to the developer and simplifies reasoning about them.
\end{abstract}

%% 2012 ACM Computing Classification System (CSS) concepts
%% Generate at 'http://dl.acm.org/ccs/ccs.cfm'.
\begin{CCSXML}
<ccs2012>
<concept>
<concept_id>10011007.10011006.10011008</concept_id>
<concept_desc>Software and its engineering~General programming languages</concept_desc>
<concept_significance>500</concept_significance>
</concept>
<concept>
<concept_id>10003456.10003457.10003521.10003525</concept_id>
<concept_desc>Social and professional topics~History of programming languages</concept_desc>
<concept_significance>300</concept_significance>
</concept>
</ccs2012>
\end{CCSXML}

\ccsdesc[500]{Software and its engineering~General programming languages}
\ccsdesc[300]{Social and professional topics~History of programming languages}
%% End of generated code

%% Keywords
%% comma separated list
\keywords{Type Systems, High-level synthesis, FPGAs}  %% \keywords are mandatory in final camera-ready submission

%% \maketitle
%% Note: \maketitle command must come after title commands, author
%% commands, abstract environment, Computing Classification System
%% environment and commands, and keywords command.
\maketitle

\section{Introduction}
Introduction goes here.


\section{Related Works}

\subsection{Halide}
Halide~\cite{halide} is a domain specific language (DSL) that provides high-level constructs for writing image processing
programs. It separates the high-level structure of an algorithm from the
performance trade-offs such as caching, loop fusion, etc.\ and automatically to
infer optimal schedules for them. \sys{} on the other provides lower-level constructs
that expose the resource constraints imposed by the hardware. \sys{} can act as a
compilation target for compiling DSLs like Halide to hardware.

There have also been efforts to compile Halide to hardware~\cite{halide-hls} using ad-hoc
restrictions on the language. \sys{} provides a more principled approach for
exposing these ad-hoc constraints.

\subsection{Spatial}
Spatial~\cite{spatial} is a domain specific programming language embedded in Scala
and compiles to Chisel. It introduces several new abstractions for extracting
parallelism out of accelerators. \sys{} is not Spatial because \ldots

\rn{Similar question. Why not build higher level primitives like
spatial and leave the safety checking and compilation to the compiler?}
\xxx[as]{And a similar answer: automation like represents an implicit trade-off. You get to not worry about the details, but then the details are hidden from you. So if you find you're getting back hardware out of Spatial, you don't see it manifest in the code and you have no recourse to fix it. Instead, we do the hard work to make all the details visible at the language level. This should be better for (a) experts who want exacting control over the hardware they generate, and (b) eventually building higher-level tools that enhance automation on top of this abstraction.}

%% Acknowledgments
\begin{acks}                            %% acks environment is optional
                                        %% contents suppressed with 'anonymous'
  %% Commands \grantsponsor{<sponsorID>}{<name>}{<url>} and
  %% \grantnum[<url>]{<sponsorID>}{<number>} should be used to
  %% acknowledge financial support and will be used by metadata
  %% extraction tools.
  This material is based upon work supported by the
  \grantsponsor{GS100000001}{National Science
    Foundation}{http://dx.doi.org/10.13039/100000001} under Grant
  No.~\grantnum{GS100000001}{nnnnnnn} and Grant
  No.~\grantnum{GS100000001}{mmmmmmm}.  Any opinions, findings, and
  conclusions or recommendations expressed in this material are those
  of the author and do not necessarily reflect the views of the
  National Science Foundation.
\end{acks}

%% Bibliography
\bibliography{./bib/venues,./bib/papers}

%% Appendix
\appendix
\section{Appendix}

Text of appendix

\end{document}
